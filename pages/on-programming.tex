% Created 2018-10-30 Tue 01:27
% Intended LaTeX compiler: pdflatex
\documentclass[11pt]{article}
\usepackage[utf8]{inputenc}
\usepackage[T1]{fontenc}
\usepackage{graphicx}
\usepackage{grffile}
\usepackage{longtable}
\usepackage{wrapfig}
\usepackage{rotating}
\usepackage[normalem]{ulem}
\usepackage{amsmath}
\usepackage{textcomp}
\usepackage{amssymb}
\usepackage{capt-of}
\usepackage{hyperref}
\author{Yuhao Zhang}
\date{\today}
\title{编程经验}
\hypersetup{
 pdfauthor={Yuhao Zhang},
 pdftitle={编程经验},
 pdfkeywords={},
 pdfsubject={},
 pdfcreator={Emacs 26.1 (Org mode 9.1.9)}, 
 pdflang={Zh-Cn}}
\begin{document}

\maketitle
\tableofcontents


\section{减少犯错几率}
\label{sec:orgfb306ef}

\section{提高代码可读性}
\label{sec:org911d662}
\subsection{辅助参数名}
\label{sec:org7cd6922}
如在Swift这样的语言中能够为参数加上\href{https://docs.swift.org/swift-book/LanguageGuide/Functions.html\#ID166}{参数标签(argument label)},功能顾名思义,就是给函数的参数增加一个可读标签。这个功能类似于Racket等语言中提供的\href{https://docs.racket-lang.org/guide/application.html\#\%2528part.\_keyword-args\%2529}{keyword}。在库函数中使用argument label的主要作用还是提高客户端代码,也就是函数调用者的代码的可读性。考虑以下代码:

\begin{verbatim}
#lang racket
(require 2htdp/image)

;; (star-polygon side-length	 	 	 	 
;;               side-count	 	 	 	 
;;               step-count
;;               outline-mode	 	 	 	 
;;               pen-or-color) → image?
;;   side-length : (and/c real? (not/c negative?))
;;   side-count : side-count?
;;   step-count : step-count?
;;   outline-mode : (or/c 'outline "outline")
;;   pen-or-color : (or/c pen? image-color?)
(star-polygon 40 7 3 "outline" "darkred")
\end{verbatim}

你能够不读函数的签名以及更多的注释信息来明白这个函数究竟是干什么的吗?很难,甚至可以说不能。在这些情况下,给参数添加辅助参数名能够为其可读性增加很好的效果:

\begin{verbatim}
(star-ploygon #:side-length 40
	      #:side-count 7
	      #:step-count 3
	      "outline" "darked")
\end{verbatim}

注意对参数断行的使用让keyword暴露出来,这是一个需要注意的技巧。

label这样的构造虽然从表面上来看并没有什么坏处,但是滥用也容易导致过度工程。由于当下IDE的广泛流行,写好文档说明以让程序员能够更快地查看文档有时往往比叠加label更加重要。因为不管你写的label能够让函数看起来多像自然语言,理解起来多容易,作为函数调用者的程序员都必须先清楚地知道这个函数的意义,也就是一般来说他们都至少会看过一次文档,调用者才能更好地完成任务。

\begin{verbatim}
struct Posn {
    var x: Int
    var y: Int
}

enum Direction {
    case north, south, east, west
}

let position = Posn(x: 0, y: 1)

/// This function moves a position Posn
func move(to direction: Direction, distance: Int) {
    position.x = position.x + distance
}
\end{verbatim}

Keyword相较于Label最大的一个不同点在于其参数是基于Keyword的位置确定的,因此函数调用者的代码可以较为灵活。

\begin{verbatim}
(define greet
  (lambda (given #:last surname)
    (string-append "Hello, " given " " surname)))

(greet "John" #:last "Smith")
; "Hello, John Smith"
(greet #:last "Doe" "John")
; "Hello, John Doe"
\end{verbatim}

但允许函数调用者代码随意放置keyword参数的位置是否会牺牲一定可读性?我觉得在不大量混用keyword和普通参数的情况下并不会有这个问题。调用者只需要注意适时断行,这样可以让keyword所在的位置一目了然。
\end{document}
