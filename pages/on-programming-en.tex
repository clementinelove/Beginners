% Created 2018-10-30 Tue 02:55
% Intended LaTeX compiler: pdflatex
\documentclass[11pt]{article}
\usepackage[utf8]{inputenc}
\usepackage[T1]{fontenc}
\usepackage{graphicx}
\usepackage{grffile}
\usepackage{longtable}
\usepackage{wrapfig}
\usepackage{rotating}
\usepackage[normalem]{ulem}
\usepackage{amsmath}
\usepackage{textcomp}
\usepackage{amssymb}
\usepackage{capt-of}
\usepackage{hyperref}
\author{Yuhao Zhang}
\date{\today}
\title{Programming Experiences}
\hypersetup{
 pdfauthor={Yuhao Zhang},
 pdftitle={Programming Experiences},
 pdfkeywords={},
 pdfsubject={},
 pdfcreator={Emacs 26.1 (Org mode 9.1.9)}, 
 pdflang={English}}
\begin{document}

\maketitle
\tableofcontents


\section{Say Goodbye to Error Prone}
\label{sec:orgfb2b26c}

\section{Increase the Code Readability}
\label{sec:orgc1bb8de}
\subsection{Argument Labels and Keyword}
\label{sec:org62a6366}

In a language like Swift, we can add argument labels to arguments in our function. The argument label in Swift works similarly to the \href{https://docs.racket-lang.org/guide/application.html\#\%2528part.\_keyword-args\%2529}{keyword} facility in Racket.

To see what argument labels and keyword could achieve for us, see the code below:

\begin{verbatim}
#lang racket
(require 2htdp/image)

;; (star-polygon side-length	 	 	 	 
;;               side-count	 	 	 	 
;;               step-count
;;               outline-mode	 	 	 	 
;;               pen-or-color) → image?
;;   side-length : (and/c real? (not/c negative?))
;;   side-count : side-count?
;;   step-count : step-count?
;;   outline-mode : (or/c 'outline "outline")
;;   pen-or-color : (or/c pen? image-color?)
(star-polygon 40 7 3 "outline" "darkred")
\end{verbatim}

Can you understand what does the call of function do without looking at the function signature of it? I guess you would tend to feel very hard to guess what does these function do. In this case, use argument labels or keywords would increate the readability in a very fancy way:

\begin{verbatim}
(star-ploygon #:side-length 40
	      #:side-count 7
	      #:step-count 3
	      "outline" "darked")
\end{verbatim}

Notice here we use line break to expose the keyword arguments; this is a technique worth noticing.

Though constructs like label seems doesn't do any bad, it can cause over-engineering if it was overly used. On the otherside we always have doc comments. Since IDE is so popular in recent days, it could be more important to write a good documentation instead of focus your mind thinking a verb for the use of an argument label. Every programmer who wants to be a 'function caller' needs to take their time reading docs and comments, especially when you encountered a function like \texttt{star-polygon}.

\begin{verbatim}
struct Posn {
    var x: Int
    var y: Int
}

enum Direction {
    case north, south, east, west
}

let position = Posn(x: 0, y: 1)

/// This function moves a position Posn
func move(to direction: Direction, distance: Int) {
    position.x = position.x + distance
}
\end{verbatim}

Keyword相较于Label最大的一个不同点在于其参数是基于Keyword的位置确定的,因此函数调用者的代码可以较为灵活。

\begin{verbatim}
(define greet
  (lambda (given #:last surname)
    (string-append "Hello, " given " " surname)))

(greet "John" #:last "Smith")
; "Hello, John Smith"
(greet #:last "Doe" "John")
; "Hello, John Doe"
\end{verbatim}

但允许函数调用者代码随意放置keyword参数的位置是否会牺牲一定可读性?我觉得在不大量混用keyword和普通参数的情况下并不会有这个问题。调用者只需要注意适时断行,这样可以让keyword所在的位置一目了然。
\end{document}
